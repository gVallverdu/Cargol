% * * * * * * * * * * * * * * * * * * * * * * * * * * * * * * * * * * * * * * *
% *
% * fichier entete de cargol.tex
% *
% * * * * * * * * * * * * * * * * * * * * * * * * * * * * * * * * * * * * * * *


% * * * * * * * * * * * * * * * * * * * * * * * * * * * * * * * * * * * * * * *
% * 
% * POLICE, ENCODAGE, LANGUE, FIGURES, COULEURS
% * 
% * * * * * * * * * * * * * * * * * * * * * * * * * * * * * * * * * * * * * * *

\usepackage[utf8]{inputenc}
\usepackage[T1]{fontenc}
\usepackage[francais]{babel}

\usepackage[svgnames]{xcolor}
\usepackage{graphicx,rotating,float,tikz}

\definecolor{turquoise}{rgb}{0 0.41 0.41}
\definecolor{rouge}{rgb}{0.79 0.0 0.1}
\definecolor{vert}{rgb}{0.15 0.4 0.1}
\definecolor{mauve}{rgb}{0.6 0.4 0.8}
\definecolor{violet}{rgb}{0.58 0. 0.41}
\definecolor{orange}{rgb}{0.8 0.4 0.2}
\definecolor{bleu}{rgb}{0.39 0.58 0.93}

\usepackage{url}

% * * * * * * * * * * * * * * * * * * * * * * * * * * * * * * * * * * * * * * *
% * 
% * MATHEMATIQUES
% * 
% * * * * * * * * * * * * * * * * * * * * * * * * * * * * * * * * * * * * * * *

\usepackage{amsmath, amsfonts, amssymb}

\setlength{\mathindent}{5ex} % position equation

% * * * * * * * * * * * * * * * * * * * * * * * * * * * * * * * * * * * * * * *
% * 
% * FORMAT DES PARAGRAPHE, ELARGISSEMENT DES TABLEAUX
% * 
% * * * * * * * * * * * * * * * * * * * * * * * * * * * * * * * * * * * * * * *

\setlength{\parskip}{1.5ex plus 0.5ex minus 0.5ex}	% espacement entre les paragraphes
\setlength{\parindent}{0em}        			% pas d'indentation
\usepackage{setspace}              			% pour gérer l'interligne

\usepackage{supertabular}
\usepackage{tabularx}

\renewcommand{\arraystretch}{1.4}  			% élargir les tableaux

% * * * * * * * * * * * * * * * * * * * * * * * * * * * * * * * * * * * * * * *
% * 
% * FOOTNOTE
% * 
% * * * * * * * * * * * * * * * * * * * * * * * * * * * * * * * * * * * * * * *

\renewcommand{\thefootnote}{\alph{footnote}}  
\newcounter{compteur}[page] 				% compteur asservis
\setcounter{compteur}{0}   
\newcommand{\myfoot}[1]{\hspace{0.5ex}\stepcounter{compteur}\footnote[\thecompteur]{#1}}
\renewcommand{\footnoterule}{\vspace{0.1cm}\rule{0.45\textwidth}{0.5pt}\vspace{0.5mm}}

% * * * * * * * * * * * * * * * * * * * * * * * * * * * * * * * * * * * * * * *
% * 
% * HYPERREF
% * 
% * * * * * * * * * * * * * * * * * * * * * * * * * * * * * * * * * * * * * * *

\usepackage{hyperref}

\hypersetup{% 
pdftex,%			Sets up hyperref for use with the pdftex program
colorlinks=true,% 		active les liens
citecolor=RoyalBlue,
linkcolor=RoyalBlue,
pdfborder={0 0 0},%		bordure des liens
pdfstartpage=1,%		page affichee a l'ouverture du pdf
pdfauthor={Germain Vallverdu},%
pdftitle={Code séquentiel de simulation de fluide simple},% 
pdfcreator={PDFLaTeX},%
pdfproducer={PDFLaTeX},%
}

% * * * * * * * * * * * * * * * * * * * * * * * * * * * * * * * * * * * * * * *
% * 
% * NOUVELLES COMMANDES
% * 
% * * * * * * * * * * * * * * * * * * * * * * * * * * * * * * * * * * * * * * *

\usepackage{xspace}
\newcommand{\eme}{$^{\text{ème}}$\xspace}%ième
\newcommand{\er}{$^{\text{er}}$\xspace}		%premier
\newcommand{\ere}{$^{\text{ère}}$\xspace}%première
\renewcommand{\degre}{$^{\circ}$\xspace}		% degre
\newcommand{\na}{\mathcal{N}_{a}\xspace}	% avogadro
\newcommand{\kt}{k_{\textrm{B}}T\xspace}	% kBT

\setlength\fboxsep{0pt}					% séparation text - contour
\setlength\fboxrule{1.5pt}				% eppaisseur du trait

% pour encadrer dans des equations
\newcommand{\encq}[2]{%
\tikz[baseline]{\node[rectangle,anchor=base,rounded corners,fill=#1]{#2};}}

% pour encadrer du texte dans un cadre rouge
\newcommand{\enc}[1]{%
\begin{center}
	\tikz{\node[rectangle,anchor=base,rounded corners,draw=rouge,thick,inner sep=2mm]{
		\parbox{0.9\textwidth}{ #1 }};}
\end{center}
}

\newcommand{\kw}[1]{ \textbf{\textsf{#1}} }

\newfloat{codesource}{htbp}{src}[chapter]
	% nom de mon nouvel environnement
	% htbp sont les options de placement de mon flottant
	% extension du fichier utilise pour construire la liste des flottants
	% niveau duquel dependra la numerotation des flottants
\floatname{codesource}{Code source} % titre du caption

% * * * * * * * * * * * * * * * * * * * * * * * * * * * * * * * * * * * * * * *
% * 
% * DEFINITION DU TITRE
% * 
% * * * * * * * * * * * * * * * * * * * * * * * * * * * * * * * * * * * * * * *

% * * * * * un titre simple aligné à droite dans une page
\makeatletter
\renewcommand{\maketitle}{
	\rule{\textwidth}{1pt}
	\vspace*{-8mm}
	\begin{flushright}
		\LARGE \bfseries \@title  \\
		\vspace*{5mm}
		\large \bfseries \@author \\
		\vspace*{4mm}
		\large \bfseries \@date \\
		\vspace*{-3mm}
	\end{flushright}
	\rule{\textwidth}{1pt} \\
}
\makeatother

% * * * * * * * * * * * * * * * * * * * * * * * * * * * * * * * * * * * 
% *
% * format de la page
% *
% * * * * * * * * * * * * * * * * * * * * * * * * * * * * * * * * * * *

\usepackage[nohead,top=2.5cm,bottom=2.5cm,left=2.5cm,right=2.5cm]{geometry}

\usepackage{fancyhdr}
\pagestyle{fancy}
\fancyhead{}                         % supprime toutes les entetes
\fancyfoot{}                         % supprime tous les pieds de page
\fancyfoot[LE,RO]{\footnotesize\thepage}          % le numero de page en bas a droite ou gauche
\fancyfoot[LO,RE]{\footnotesize\sffamily\itshape Cargol : G. Vallverdu}
\renewcommand{\headrulewidth}{0pt}
\renewcommand{\footrulewidth}{0pt}

% redefini le syle plain
\fancypagestyle{plain}{%
\fancyhf{}                           % clear all header and footer fields
\fancyfoot[RE,RO]{\thepage}
\renewcommand{\headrulewidth}{0pt}
\renewcommand{\footrulewidth}{0pt}}

% * * * * * * * * * * * * * * * * * * * * * * * * * * * * * * * * * * * 
% *
% * package listing
% *
% * * * * * * * * * * * * * * * * * * * * * * * * * * * * * * * * * * *

\usepackage{listings}
\lstset{% general command to set parameter(s)
language=C,                                     % langage
identifierstyle=,                               % rien
basicstyle=\footnotesize\sffamily,              % style du texte
stringstyle=\sffamily\color{green},             % styme typewriter pour string
keywordstyle=\color{DarkRed}\bfseries,        % style des mots clef
commentstyle=\color{RoyalBlue},                      % style commentaire
backgroundcolor=\color{black!3},                  % couleur arrière plan
}

% * * * * * * * * * * * * * * * * * * * * * * * * * * * * * * * * * * * 
% *
% * format titres et sections
% *
% * * * * * * * * * * * * * * * * * * * * * * * * * * * * * * * * * * *

\usepackage{titlesec}
% \titleformat{ command }[ shape ]{ format }{ label }{ sep }{ before }[ after ]
% \titlespacing*{ command }{ left }{ dessus }{ apres }

\titleformat{\chapter}[block]%
{\LARGE\bfseries\sffamily\filleft}%	% format
{\titlerule \\ \thechapter}%		% label
{2ex}%					% séparation label
{}%					% avant = (entre label et texte)
[\vspace{1ex} \titlerule]		% après
\titlespacing*{\chapter}{0cm}{0cm}{1cm}

\newcounter{gsec}[chapter]
\setcounter{gsec}{0}
\titleformat{\section}{\normalfont\Large\bfseries}{\stepcounter{gsec}\thegsec}{1em}{}
\titlespacing*{\section}{0ex}{3.5ex plus 1ex minus 0.2ex}{2.3ex plus .2ex}

\newcounter{gsubsec}[section]
\setcounter{gsubsec}{0}
\titleformat{\subsection}{\normalfont\large\bfseries}{\stepcounter{gsubsec}\thegsec.\thegsubsec}{1em}{}
\titlespacing*{\subsection}{0ex}{3.25ex plus 1ex minus 0.2ex}{1.5ex plus .2ex}

\titleformat{\subsubsection}[hang]{\bfseries}{}{0ex}{}
\titlespacing*{\subsubsection}{0ex}{2ex}{0ex}

%\usepackage{titletoc}
%\titlecontents{section}[10ex]{}{\sffamily\bfseries\Large\contentslabel{4ex}}{}
%{\sffamily\titlerule*[0.5pc]{.}\contentspage}[\vspace*{5ex}]

\usepackage{fncylab}
\labelformat{figure}{\sffamily \figurename~#1}
\labelformat{table}{\tablename~#1}
\labelformat{equation}{\sffamily équation~#1}
\labelformat{codesource}{\sffamily code source~#1}

% * * * * * * * * * * * * * * * * * * * * * * * * * * * * * * * * * * *